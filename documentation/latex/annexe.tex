\section{\textit{Annexe à destination des encadrants}}

\subsection{Matériel}
\label{sec:matériel}

Voici la liste du matériel utilisé pour ce TP. Des produits similaires peuvent
être utilisés, vous pouvez vous référer à la section \ref{sec:références} pour
comprendre les spécificités du matériel.

\begin{itemize}
    \item \textbf{Raspberry Pi 5} 4 Go BCM2712 2,4 GHz \url{https://fr.rs-online.com/web/p/raspberry-pi/0219253}
    \item \textbf{Carte mémoire} A2 class micro-SD cards for DDR50 and SDR104 32GB  \url{https://mou.sr/3Km0bGc} 
    \item \textbf{Alimentation pour Raspberry Pi} 1.2m USB type C avec Fiche femelle européenne \url{https://fr.rs-online.com/web/p/alimentations-raspberry-pi/0219261}
    \item \textbf{Adafruit NeoPixel} Digital RGBW LED Strip - White PCB 60 LED/m 1m \url{https://mou.sr/40tlNWa} 
    \item \textbf{Female DC Power Adapter} \url{https://mou.sr/3NrRk6Y}
    \item \textbf{Adaptateur AC/DC} 5V c.c., 4A, 24W, C14 \url{https://fr.rs-online.com/web/p/adaptateurs-ac-dc/9048474} 
    \item \textbf{Cordon d'alimentation RS PRO} Connecteur CEI C13 CEE 7/7, 1m, 10 A / 250 V \url{https://fr.rs-online.com/web/p/cordons-d-alimentation/6266751} 
    \item \textbf{TXB0104 Bi-Direction Level Shifter} \url{https://mou.sr/4heobXq} 
    \item \textbf{HDMI vers micro HDMI} 1 m \url{https://fr.rs-online.com/web/p/cables-raspberry-pi/2012171} 
    \item \textbf{Platine d'essai} \url{https://fr.rs-online.com/web/p/platines-d-essais/2153175}
    \item \textbf{Strap pour platine d'essai mâle-mâle} 
    \begin{sloppypar}
    \url{https://fr.rs-online.com/web/p/straps-pour-platines-d-essai/2048241}
    \end{sloppypar}
    \item \textbf{Strap pour platine d'essai mâle-femelle} 
    \begin{sloppypar}
    \url{https://fr.rs-online.com/web/p/straps-pour-platines-d-essai/2048243}
    \end{sloppypar}
\end{itemize}

Si vous souhaitez réaliser la première séance avec une LED unique, il en faudra également une,
ainsi qu'une résistance. \par

\vspace{1em}

\textit{\textbf{Note:}} Si vous utilisé le convertisseur de niveau (Level-Shifter) proposé ici, il faudra souder
les broches fournies avec vous même. 

\subsection{Configuration du Raspberry Pi}

Dans un premier temps, il faut installer Raspberry Pi OS sur la carte mémoire et configurer le Raspberry Pi 5. Pour simplifier cette étape,
vous trouverez dans ce Proton Drive une image de la carte mémoire déjà configurée. 

\vspace{1em}

\textit{\textbf{Lien vers l'image de la carte SD du RPi5:}} 
\begin{sloppypar}
\url{https://drive.proton.me/urls/3HZFCYNRZ4#hk5mEik54Qc6}
\end{sloppypar}

\vspace{1em}

Pour décompresser l'image, utilisez la commande:
\color{blue} \texttt{tar -xvzf rpi5-illumine.tar.gz} \color{black}
Puis voici une référence pour vous aider à la copier sur votre carte SD: 
\begin{sloppypar}
\url{https://raspberrytips.fr/cloner-carte-sd-raspberry-pi/}
\end{sloppypar}

\vspace{1em}

L'environnement virtuel  python \color{blue} \texttt{illumine\_env} \color{black} est déjà créé et contient toutes les librairies nécessaires, mais voici 
la liste de celles assez spécifiques si vous souhaitez le refaire vous même:
\begin{itemize} \color{blue}
    \item gpiozero
    \item Adafruit-Blinka-Raspberry-Pi5-Neopixel
    \item adafruit-circuitpython-pixelbuf
    \item adafruit-circuitpython-led-animation
    \item ipython
\end{itemize} \color{black}
En plus, vous aurez besoin de librairies plus classiques comme \color{blue}\texttt{numpy}, \texttt{time} \color{black}, etc.

\vspace{1em}

\textit{\textbf{Note:}} Si vous modifiez les chemins des dossiers, n'oubliez pas de modifier le TP en conséquence (commande \color{blue} \texttt{cd} \color{black}).

\subsection{Références et ressources}
\label{sec:références}

Toutes les explications claires et détaillées sont disponibles sur le merveilleux site d'Adafruit.
\begin{itemize}
    \item Une présentation des rubans LED NeoPixel: 
    \begin{sloppypar}
    \url{https://learn.adafruit.com/neopixels-on-raspberry-pi}
    \end{sloppypar}
    \item Comment réaliser le montage électronique: 
    \begin{sloppypar}
    \url{https://learn.adafruit.com/neopixels-on-raspberry-pi/raspberry-pi-wiring}
    \end{sloppypar}
    \item Les librairies python à installer et un exemple de code: 
    \begin{sloppypar}
    \url{https://learn.adafruit.com/neopixels-on-raspberry-pi/python-usage}
    \end{sloppypar}
    \item Les modifications nécessaires pour le Raspberry Pi 5: 
    \begin{sloppypar}
    \url{https://learn.adafruit.com/circuitpython-on-raspberrypi-linux/using-neopixels-on-the-pi-5}
    \end{sloppypar}
    \item Présentation du Level Shifter TXB0104: 
    \begin{sloppypar}
    \url{https://www.adafruit.com/product/1875}
    \end{sloppypar}
\end{itemize}

Si vous souhaitez réaliser le première séance avec une LED unique, voici des ressources utiles:
\begin{itemize}
    \item Le circuit électronique simple 
    \begin{sloppypar}
    \url{https://raspberry-pi.fr/led-raspberry-pi/}
    \end{sloppypar}
    \item La librairie gpiozero 
    \begin{sloppypar}
    \url{https://raspberrypi.stackexchange.com/questions/148686/raspberry-pi-5-gpiozero}
    \end{sloppypar}
\end{itemize}