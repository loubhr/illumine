\section{\textit{Deuxième Séance}}

\subsection{Montage électronique}
On va connecter le ruban LED au raspberry Pi, en utilisant une carte de prototypage, en suivant le schéma ci-dessous:

\begin{figure}[htbp]
    \centering
    \includegraphics[width=\textwidth]{Figures/LED_tape_circuit.png}
    \caption{Schéma de connexion du ruban LED au Raspberry Pi}
    \label{fig:schema_led_tape_rpi}
\end{figure}

\subsection{Programmation Python}

\subsubsection{Démarrage}

Une fois le montage électronique réalisé, on peut brancher le clavier, la souris, et le câble micro-hdmi au Raspberry Pi.
Puis, on peut brancher le câble d'alimentation pour allumer le Raspberry Pi 5.

Il va s'allumer, ou "booter", et afficher l'écran de sélection de l'utilisateur.
On va sélectionner:
\begin{itemize}
    \item Utilisateur : \textbf{illumine}
    \item Mot de passe : \textbf{illumine}
\end{itemize}

Après quelques secondes, le bureau du Raspberry Pi 5 va s'afficher.
On va ouvrir le terminal (icône noire avec un \textgreater\_ blanc) pour pouvoir taper des commandes,
comme montré sur la figure \ref{fig:rpi_desktop}.

\begin{figure}[ht]
    \centering
    \includegraphics[width=0.8\textwidth]{./Figures/démarrage.png}
    \caption{Bureau du Raspberry Pi 5}
    \label{fig:rpi_desktop}
\end{figure}

\subsubsection{Commandes python}

On va se placer dans un environnement python interactif en tapant les commandes suivantes dans le terminal:
\color{blue}\begin{verbatim}
    cd rpi5_projects/git_repos/illumine/python/2eme_seance/
    source_illumine_env
    ipython
\end{verbatim}
\color{black}

On est maintenant dans un environnement Python interactif qui ressemble à celui sur la figure \ref{fig:ipython}.

\begin{figure}[ht]
    \centering
    \includegraphics[width=0.8\textwidth]{./Figures/ipython.png}
    \caption{Environnement Python interactif}
    \label{fig:ipython}
\end{figure}

On peut donc taper des commandes python pour contôler le ruban LED:
\color{blue}\begin{verbatim}
    import RPi5_Neopixel as np 
    ruban = np.Neopixel_Tape()
    ruban.on_all_led('orchid')
\end{verbatim}
\color{black}

On remarque qu'il faut donner un nom de couleur en anglais, alors voici la liste des couleurs disponibles:

\color{blue}
\begin{multicols}{3}
\begin{verbatim}
    'red'
    'dark_red'
    'red_orange'
    'crimson'
    'rust'
    'blood_red'
    'tomato'
    'reddish_orange'
    'burnt_orange'
    'pumpkin'
    'orange'
    'dark_orange'
    'golden_orange'
    'yellow_orange'
    'yellow'
    'bright_yellow'
    'lemon_yellow'
    'pale_yellow'
    'lime_yellow'
    'chartreuse'
    'spring_green'
    'light_green'
    'green'
    'dark_green'
    'forest_green'
    'teal_green'
    'teal'
    'bright_turquoise'
    'medium_turquoise'
    'dark_turquoise'
    'cyan'
    'light_cyan'
    'aqua'
    'light_sky_blue'
    'cornflower_blue'
    'steel_blue'
    'dodger_blue'
    'light_blue'
    'powder_blue'
    'cadet_blue'
    'lavender_blue'
    'light_indigo'
    'indigo'
    'blue_violet'
    'purple'
    'medium_purple'
    'orchid'
    'violet'
    'thistle_purple'
    'pink_purple'
    'magenta'
    'fuchsia'
    'hot_pink'
    'light_pink'
    'pink'
    'light_coral'
    'salmon_pink'
    'coral'
    'tomato_pink'
    'rosy_brown'
\end{verbatim}
\end{multicols}
\color{black}

N'oublie pas les guillements autour du nom de la couleur !

\subsubsection{Fonctions élémentaires}

Plusieurs fonctions élémentaires sont disponibles pour piloter une ou plusieurs LED du ruban. \par

\textit{Exercice:} Essaye les fonctions suivantes en remplaçant \textbf{num} par un numéro de LED entre 0 et 59,
\textbf{color\_name} par un nom de couleur de la liste précédente, et \textbf{times} par le nombre de fois
que tu veux faire l'effet.
Décris sur les pointillés ce que fait chaque fonction.

\color{blue}\begin{verbatim}
    ruban.on_led(num, color_name): ............................................... 

    ruban.off_led(num): ..........................................................

    ruban.on_all_led(color_name): ................................................

    ruban.off_all_led(): .........................................................

    ruban.avance(num, color_name, times): ........................................

    ruban.recule(num, color_name, times): ........................................

    ruban.clignote(num, color_name, times): ......................................

    ruban.clignote_all(color_name, times): .......................................

    ruban.random(color_name, times): ............................................

    ruban.fill_rainbow(): ........................................................
\end{verbatim}
\color{black}

\subsubsection{Script python}

Nous allons maintenant écrire des fonctions plus complexes, mais pour cela nous devons rapidement expliquer
ce que fait une boucle \textit{while}.
Une boucle \textit{while} permet de répéter une série d'instructions tant qu'une condition est vraie.
Une analogie peut être "tant que je ne suis pas arrivée au bout du chemin, je continue d'avancer".
Voici un exemple de boucle \textit{while} en python (à ne pas taper, c'est juste un exemple):
\color{blue}
\begin{verbatim}
    compteur = 0
    while compteur < 5:
        compteur = compteur + 1
\end{verbatim}
\color{black}

On va maintenant quitter l'environnement python interactif en tapant la commande \color{blue}
\texttt{exit()} \color{black}.

\textit{Exercice:} Tape la commande suivante pour ouvrir et compléter le script python \texttt{charge.py}:
\color{blue}
\begin{verbatim}
    gedit charge.py
\end{verbatim}
\color{black}

Une fenêtre comme celle de la figure \ref{fig:charge} va s'ouvrir. La description de la fonction est donnée, 
et c'est à toi de compléter le script en utilisant les fonctions élémentaires vues précédemment, ainsi qu'une
ou plusieurs boucles \textit{while} si nécessaire (et/ou \textit{for}).

\begin{figure}[ht]
    \centering
    \includegraphics[width=0.8\textwidth]{./Figures/fonction_à_compléter.png}
    \caption{Fonction \texttt{charge.py} à compléter}
    \label{fig:charge}
\end{figure}

Une fois terminé, sauvegarde le fichier (Ctrl+S), et tu peux soit fermer la fenêtre et revenir au terminal, soit
garder la fenêtre ouverte, ouvrir un autre terminal et retaper:
\color{blue} 
\begin{verbatim} 
    cd rpi5_projects/git_repos/illumine_ton_ingeniosite/2nd_seance/
    source_illumine_env
\end{verbatim}
\color{black} 

Dans les deux cas, tape:
\color{blue}
\begin{verbatim}
    python3 charge.py
\end{verbatim}
\color{black}
pour exécuter ton script python et voir si le résultat est celui attendu !

Refait cela en remplaçant la fonction \texttt{charge.py} par les fonctions \texttt{chariot.py}
et \texttt{croise.py}.

\subsubsection{Eteindre le RPi 5}

On peut éteindre le RPi 5 comme un ordinateur classique, mais on peut aussi l'éteindre avec
une commande dans le terminal:

\color{blue}\begin{verbatim}
    sudo shutdown -h now
\end{verbatim}

\color{black}

Il faut taper le mot de passe: rien ne s'affiche quand tu tapes mais c'est normal !
Tape le bon mot de passe et le RPi 5 va s'éteindre.

Bravo !



